\chapter{Freenet}
\section{FCPv2 protocol}
Freenet uses a simple, text-based protocol called FCPv2 (Freenet Client Protocol) to allow third-party applications to interact with Freenet. \newline
Functionalities supported by the FCPv2 Protocol are:\cite{FCP}
\begin{itemize}
    \item Inserting of data into Freenet
    \item Retrieval of data from Freenet
    \item Querying the status of Freenet
    \item Managing the other Freenet nodes that are connected to your own node.
\end{itemize}
\newpage
\subsection{ClientHello}
The first message sent from the third-party client to the node on a given connection is always the ClientHello message. No other messages are allowed to be sent to the node before the ClientHello is sent. The node will always respond with a NodeHello message.
\begin{verbatim}
    ClientHello
    Name=My Freenet Client
    ExpectedVersion=2.0
    EndMessage
\end{verbatim}
\begin{tabularx}{\textwidth}{p{3cm}|p{2cm}|p{2.5cm}|p{5.5cm}}
\toprule
    Field & Possible values & Mandatory? & Description \\
\midrule 
Name & Arbitrary string, on one line & yes & A name to uniquely identify the client to the node. \\
\addlinespace
\hline
\addlinespace
ExpectedVersion & 2.0 & yes & The version of FCP to expect. In this case it will always be 2.0 \\
\bottomrule
\end{tabularx}
\newpage
\subsection{NodeHello}
The NodeHello message is the reply a node sends after receiving a ClientHello message from a client.
\begin{verbatim}
    NodeHello
    CompressionCodecs=4 - GZIP(0), BZIP2(1), LZMA(2), LZMA_NEW(3)
    Revision=build01475
    Testnet=false
    Version=Fred,0.7,1.0,1475
    Build=1475
    ConnectionIdentifier=6f467be43d838f8e02877e7f176a73bd
    Node=Fred
    ExtBuild=29
    FCPVersion=2.0
    NodeLanguage=ENGLISH
    ExtRevision=v29
    EndMessage
\end{verbatim}
\begin{tabularx}{\textwidth}{p{3cm}|p{3cm}|p{7.5cm}}
\toprule
    Field & Example or [range] & Description \\
\midrule 
FCPVersion & 2.0 &  The version of FCP being utilized for this connection. \\
\addlinespace
\hline
\addlinespace
Node & Fred & \\
\addlinespace
\hline
\addlinespace
Version & Fred, 0.7, 1.0, 1475 & A comma-delimited list consisting of 
\begin{itemize}
    \item node name
    \item node version
    \item protocol version
    \item build number
\end{itemize}\\
\addlinespace
\hline
\addlinespace
Build & 1475 & \\
\addlinespace
\hline
\addlinespace
Revision & build01475 & \\
\addlinespace
\hline
\addlinespace
ExtBuild & 29 & The build of freenet-ext.jar being utilized by the node.\\
\addlinespace
\hline
\addlinespace
ExtRevision & v29 & The revision of freenet-ext.jar being utilized by the node.\\
\addlinespace
\hline
\addlinespace
Testnet & true, false & \\
\addlinespace
\hline
\addlinespace
Compression Codecs & 4 - GZIP(0), BZIP2(1), LZMA(2), LZMA_NEW(3) & A list of the compression codecs supported by the node.\\
\addlinespace
\hline
\addlinespace
Connection Identifier & 6f467be43d838f8 e02877e7f176a 73bd & A unique connection identifier, reported in hexadecimal format.\\
\addlinespace
\hline
\addlinespace
NodeLanguage & ENGLISH & The language currently utilized by the node.\\
\bottomrule
\end{tabularx}
\subsection{ClientPut}
The ClientPut message is used to specify an insert into Freenet of a single file. This file can be inserted by referring to a file on the disk, including the data directly in the message or redirecting to another key.
\begin{verbatim}
    ClientPut
    URI=KSK@
    Metadata.ContentType=text/plain
    Identifier=My Test Insert
    DataLength=4
    Data
    <Data>
\end{verbatim}
\begin{tabularx}{\textwidth}{p{3cm}|p{2cm}|p{2.5cm}|p{5.5cm}}
\toprule
    Field & Possible values & Mandatory? & Description \\
\midrule 
URI & CHK@ KSK@ SSK@ USK@ & yes & The type of key to insert. \\
\addlinespace
\hline
\addlinespace
Metadata. ContentType & Any MIME type & no & The MIME type of the data being inserted. \\
\addlinespace
\hline
\addlinespace
Identifier & Arbitrary text string & yes & This is just for client to be able to identify files that have been inserted. \\
\addlinespace
\hline
\addlinespace
DataLength & Integer from 0 to ? & yes & The length of the data file being included in this FCP message. \\
\addlinespace
\hline
\addlinespace
Data &  &  & Indicates the end of the message. The data follows this line. \\
\bottomrule
\end{tabularx}
\newpage
\subsection{ClientGet}
The ClientGet message is used to specify a get of a single file from Freenet.
\begin{verbatim}
    ClientGet
    URI=KSK@
    Identifier=My Test Identifier
    EndMessage
\end{verbatim}
\begin{tabularx}{\textwidth}{p{3cm}|p{2cm}|p{2.5cm}|p{5.5cm}}
\toprule
    Field & Possible values & Mandatory? & Description \\
\midrule 
URI & CHK@ KSK@ SSK@ USK@ & yes & The type of key to insert. \\
\addlinespace
\hline
\addlinespace
Identifier & Arbitrary text string & yes & This is just for client to be able to identify files that have been inserted. \\
\bottomrule
\end{tabularx}
\newpage
\subsection{DataFound}
DataFound indicates a successful fetch of the key, but does not actually include the data.
\begin{verbatim}
    DataFound
    Identifier=Request Number One
    Metadata.ContentType=text/plain;charset=utf-8
    DataLength=37261
    EndMessage
\end{verbatim}
\begin{tabularx}{\textwidth}{p{3cm}|p{3cm}|p{7.5cm}}
\toprule
    Field & Example or [range] & Description \\
\midrule 
Identifier & Arbitrary text string & This is just for client to be able to identify files that have been inserted. \\
\addlinespace
\hline
\addlinespace
Metadata. ContentType & Any MIME type & The MIME type of the data being inserted. \\
\addlinespace
\hline
\addlinespace
DataLength & Integer from 0 to ? & The length of the data file being included in this FCP message. \\
\bottomrule
\end{tabularx}
\newpage
\subsection{AllData}
AllData is a message from the node returning data from the node.
\begin{verbatim}
    AllData
    Identifier=Request Number One
    DataLength=37261 // length of data
    StartupTime=1189683889
    CompletionTime=1189683889
    Metadata.ContentType=text/plain;charset=utf-8
    Data
     <37261 bytes of data>
\end{verbatim}
\begin{tabularx}{\textwidth}{p{3cm}|p{3cm}|p{7.5cm}}
\toprule
    Field & Example or [range] & Description \\
\midrule 
Identifier & Arbitrary text string & This is just for client to be able to identify files that have been inserted. \\
\addlinespace
\hline
\addlinespace
DataLength & Integer from 0 to ? & The length of the data file being included in this FCP message. \\
\addlinespace
\hline
\addlinespace
StartupTime & Time & The startup time of the GetRequest. \\
\addlinespace
\hline
\addlinespace
CompletionTime & Time & The completion time of the GetRequest. \\
\addlinespace
\hline
\addlinespace
Metadata. ContentType & Any MIME type & The MIME type of the data being inserted. \\
\bottomrule
\end{tabularx}
\newpage
\subsection{GenerateSSK}
The GenerateSSK message is used to ask the node to generate us an SSK keypair. Response will come back in a SSKKeypair message.
\begin{verbatim}
    GenerateSSK
    Identifier=My Identifier Blah Blah
    EndMessage
\end{verbatim}
\begin{tabularx}{\textwidth}{p{3cm}|p{2cm}|p{2.5cm}|p{5.5cm}}
\toprule
    Field & Possible values & Mandatory? & Description \\
\midrule 
Identifier & Arbitrary text string & yes & This is just for client to be able to identify files that have been inserted. \\
\bottomrule
\end{tabularx}
\newpage
\subsection{SSKKeypair}
Is a message sent from the Freenet node to a client program in response to the client issuing a GenerateSSK command.
\begin{verbatim}
    SSKKeypair
    InsertURI=freenet:SSK@AKTTKG6YwjrHzWo67laRcoPqibyiTdyYufjVg54fBlWr,
               AwUSJG5ZS-FDZTqnt6skTzhxQe08T-fbKXj8aEHZsXM/
    RequestURI=freenet:SSK@BnHXXv3Fa43w~~iz1tNUd~cj4OpUuDjVouOWZ5XlpX0,
               AwUSJG5ZS-FDZTqnt6skTzhxQe08T-fbKXj8aEHZsXM,AQABAAE/
    Identifier=My Identifier from GenerateSSK
    EndMessage
\end{verbatim}
\begin{tabularx}{\textwidth}{p{3cm}|p{3cm}|p{7.5cm}}
\toprule
    Field & Example or [range] & Description \\
\midrule 
InsertURI & inserting URI & Consists the URI needed to insert to freenet. \\
\addlinespace
\hline
\addlinespace
RequestURI & request URI & Consists the URI needed to request from freenet. \\
\addlinespace
\hline
\addlinespace
Identifier & Arbitrary text string & This is for the Client to Identify to witch GenerateSSK Request this Keypair is. \\
\bottomrule
\end{tabularx}
\newpage